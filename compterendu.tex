\documentclass[a4paper,11pt]{article}
\usepackage[T1]{fontenc}
\usepackage[utf8]{inputenc}
\usepackage{lmodern}
\usepackage[francais]{babel}
\usepackage{color}
\oddsidemargin=-0,8cm
\headsep=-1,5cm

\setlength{\parskip}{0,3cm}


\title{Classification de documents multi dimensionnelle simplifiée avec le modèle de Bernoulli}
\author{MARCO Florian et BARROIS Florian}
\date{}


\begin{document}
\maketitle
%\makeindex

\section{Introduction}

Dans ce projet nous allons essayer de déterminer la classification d'un ensemble de document à partir de la fréquence d'apparition comparée à celle d'une base étalon.

Exemple simplifié : si pour une classe A, nous voyons surtout certains type de document ( par exemple des documents texte), si pour une classe B, nous voyons surtout d'autres type de document ( par exemple des documents audios). 
Nous avons un ensemble de document de classe inconnue, nous pourrons donc retrouver sa classe en fonction des fréquences d'apparition.

\section{Problematique}

Nous allons chercher à représenter un document sous forme de vecteur de variables aléatoires discrètes de Bernoulli. 

\section{Parsing}

Nous allons ouvrir le fichier, et pour chaque ligne qui correspond à une classe, nous allons la stocker dans un dictionnaire

\section{Classification}

Nous faisons une premiere lecture

Nous faisons une seconde lecture

\section{Utilisation}

Pour utiliser notre programme, il suffit juste d'utiliser Python3.

\end{document}
