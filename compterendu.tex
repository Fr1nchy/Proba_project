\documentclass[a4paper,11pt]{article}
\usepackage[T1]{fontenc}
\usepackage[utf8]{inputenc}
\usepackage{lmodern}
\usepackage[francais]{babel}
\usepackage{color}
\oddsidemargin=-0,8cm
\headsep=-1,5cm

\setlength{\parskip}{0,3cm}


\title{Classification de documents multi dimensionnelle simplifiée avec le modèle de Bernoulli}
\author{MARCO Florian et BARROIS Florian}
\date{09/03/2017}


\begin{document}
\maketitle
%\makeindex

\section{Introduction}

Dans ce projet nous allons essayer de déterminer la classification d'un ensemble de document à partir de la fréquence d'apparition comparée à celle d'une base étalon.

Exemple simplifié : si pour une classe A, nous voyons surtout certains type de document ( par exemple des documents texte), si pour une classe B, nous voyons surtout d'autres type de document ( par exemple des documents audios). 
Nous avons un ensemble de document de classe inconnue, nous pourrons donc retrouver sa classe en fonction des fréquences d'apparition.

\section{Problematique}

Nous allons chercher à représenter un document sous forme de vecteur de variables aléatoires discrètes de Bernoulli. 

\section{Parsing et Utilisation}

Notre programme va commencer par parser le fichier en entrée :
Il va lire le fichier ligne par ligne, les découper à chaque espace espaces, stocker le numéro de classe dans un dictionnaire.
Il va ensuite ajouter dans le dictionnaire de cette classe une liste de dictionnaire, et pour chaque ligne de cette classe va ajouter ses données sous forme de couple nom_du_document:nombre_de_presence.
Nous faisons un tirage random entre 1 et 100, si notre tirage est inférieur à 31, nous considérons la ligne lue comme une ligne du test, et pour le reste comme une ligne de la base d'apprentissage. On compte en même temps le nombre de docs de chaque classes.  De plus nous allons stocker dans une liste l'ensemble des mots présents dans ces classes pour pouvoir repérer ultérieurement leur absence.

Pour les mots présents dans les classes de test, nous stockons uniquement ceux qu'on va voir apparaitre.

Nous faisons une seconde lecture où nous calculons à la fois la fréquence d'apparition , ou la valeur de la probabilité d'absence de chaque mot dans chaque classe.


\section{Utilisation}

Pour utiliser notre programme, il suffit juste d'utiliser Python3.

\end{document}
